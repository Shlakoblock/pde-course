\chapter{Исследование сходимости}
\section{Оценка погрешности}

Итак, мы получили решение в виде ряда. Очевидно, что подобный ряд сходится. Чтобы выдавать численное значение напряжённости, нам необходимо суммировать определённое конечное количество элементов ряда. Пусть это будет $M$ элементов.
$S_M = \sum \limits_{m=1}^{M} \left( \frac{2\pi c^2}{m(w_m^2 - k^2)}\left(\frac{4l_y^2 - \lambda^2}{l_y^2\lambda^2} \right)\sin{kt} - \frac{k}{w_m}\left(\frac{2}{\pi m} + D\right)\sin{w_mt} \right) \sin\frac{\pi y}{l_y} \sin\frac{\pi m z}{l_z} + \\
    + \frac{l_z - z}{l_z} \sin\frac{2\pi c}{\lambda}t \sin\frac{\pi y}{l_y}$

Эту частичную сумму мы и будем выдавать в качестве ответа. При этом мы ошибёмся на величину, равную модулю хвоста ряда, начиная с $(M+1)$-ого элемента. Обозначим за $r_M$ хвост нашего ряда.

$r_M = \sum \limits_{m=M}^{\infty} \left( \frac{2\pi c^2}{m(w_m^2 - k^2)}\left(\frac{4l_y^2 - \lambda^2}{l_y^2\lambda^2} \right)\sin{kt} - \frac{k}{w_m}\left(\frac{2}{\pi m} + D\right)\sin{w_mt} \right) \sin\frac{\pi y}{l_y} \sin\frac{\pi m z}{l_z}$

Оценим сверху модуль хвоста ряда $|r_{M+1}|$ и выясним, как величина ошибки зависит от количества просуммированных элементов. Воспользуемся неравенством треугольника, чтобы перейти к оценки каждого слагаемого, затем все синусы оценим сверху единицами.

$|r_{M+1}| = \left| \sum \limits_{m=M+1}^{\infty} \left( \frac{2\pi c^2}{m(w_m^2 - k^2)}\left(\frac{4l_y^2 - \lambda^2}{l_y^2\lambda^2} \right)\sin{kt} - \frac{k}{w_m}\left(\frac{2}{\pi m} + D\right)\sin{w_mt} \right) \sin\frac{\pi y}{l_y} \sin\frac{\pi m z}{l_z} \right| \leq \\
\leq \sum \limits_{m=M+1}^{\infty} \left| \left( \frac{2\pi c^2}{m(w_m^2 - k^2)}\left(\frac{4l_y^2 - \lambda^2}{l_y^2\lambda^2} \right)\sin{kt} - \frac{k}{w_m}\left(\frac{2}{\pi m} + D\right)\sin{w_mt} \right) \sin\frac{\pi y}{l_y} \sin\frac{\pi m z}{l_z} \right| = \\
= \sum \limits_{m=M+1}^{\infty} \left| \frac{2\pi c^2}{m(w_m^2 - k^2)}\left(\frac{4l_y^2 - \lambda^2}{l_y^2\lambda^2} \right)\sin{kt} - \frac{k}{w_m}\left(\frac{2}{\pi m} + D\right)\sin{w_mt} \right| \left| \sin\frac{\pi y}{l_y} \right| \left| \sin\frac{\pi m z}{l_z} \right| \leq \\
\leq \sum \limits_{m=M+1}^{\infty} \left| \frac{2\pi c^2}{m(w_m^2 - k^2)}\left(\frac{4l_y^2 - \lambda^2}{l_y^2\lambda^2} \right)\sin{kt} - \frac{k}{w_m}\left(\frac{2}{\pi m} + D\right)\sin{w_mt} \right| \leq \\
\leq \sum \limits_{m=M+1}^{\infty} \left| \frac{2\pi c^2}{m(w_m^2 - k^2)}\left(\frac{4l_y^2 - \lambda^2}{l_y^2\lambda^2} \right)\sin{kt} \right| + \sum \limits_{m=M+1}^{\infty} \left| \frac{k}{w_m}\left(\frac{2}{\pi m} + D\right)\sin{w_mt} \right| = \\
= \sum \limits_{m=M+1}^{\infty} \left| \frac{2\pi c^2}{m(w_m^2 - k^2)} \frac{4l_y^2 - \lambda^2}{l_y^2\lambda^2} \right| \left| \sin{kt} \right| + \sum \limits_{m=M+1}^{\infty} \left| \frac{k}{w_m}\left(\frac{2}{\pi m} + D\right) \right| \left| \sin{w_mt} \right| \leq \\
\leq \sum \limits_{m=M+1}^{\infty} \left| \frac{2\pi c^2}{m(w_m^2 - k^2)} \frac{4l_y^2 - \lambda^2}{l_y^2\lambda^2} \right| + \sum \limits_{m=M+1}^{\infty} \left| \frac{k}{w_m}\left(\frac{2}{\pi m} + D\right) \right| \leq \\
\leq \sum \limits_{m=M+1}^{\infty} \left| \frac{2\pi c^2}{m(w_m^2 - k^2)} \frac{4l_y^2 - \lambda^2}{l_y^2\lambda^2} \right| + \sum \limits_{m=M+1}^{\infty} \left| \frac{2 k}{\pi m w_m} \right| + \sum \limits_{m=M+1}^{\infty} \left| \frac{k D}{w_m} \right|$

Если удастся оценить каждый из этих хвостов, то сумма полученных оценок будет оценкой остатка исходного ряда. Для получения оценок восользуемся интегральным признаком Коши: \\
$\sum \limits_{n = N + 1}^{\infty} f(n) \leq \int \limits_N^{\infty} f(x) \diff x$

Cчитаем, кроме того, что $2 l_y > \lambda$, т.~е. размер стенки волновода превосходит половину длины волны.

$\sum \limits_{m=M+1}^{\infty} \left| \frac{2\pi c^2}{m(w_m^2 - k^2)} \frac{4l_y^2 - \lambda^2}{l_y^2\lambda^2} \right| =
\frac{4l_y^2 - \lambda^2}{l_y^2\lambda^2} \sum \limits_{m=M+1}^{\infty} \frac{2\pi c^2}{m \left| w_m^2 - k^2 \right|} =
\frac{4l_y^2 - \lambda^2}{l_y^2\lambda^2} \sum \limits_{m=M+1}^{\infty} \frac{2\pi c^2}{m \left| \pi^2 c^2 \left( \frac{1}{l_y^2} + \frac{m^2}{l_z^2} \right) - \frac{4 \pi^2 c^2}{\lambda^2} \right|} = \\
= \frac{4l_y^2 - \lambda^2}{l_y^2\lambda^2} \sum \limits_{m=M+1}^{\infty} \frac{2 \lambda^2 l_y^2 l_z^2}{m \pi \left| m^2 \lambda^2 l_y^2 - 4 l_y^2 l_z^2 + \lambda^2 l_z^2 \right|} =
\frac{4l_y^2 - \lambda^2}{l_y^2\lambda^2} \sum \limits_{m=M+1}^{\infty} \frac{2 l_z^2}{m \pi \left| m^2 - \frac{4 l_y^2 l_z^2 - \lambda^2 l_z^2}{\lambda^2 l_y^2} \right|} = \\
= \frac{4l_y^2 - \lambda^2}{\lambda^2} \frac{l_z^2}{l_y^2} \sum \limits_{m=M+1}^{\infty} \frac{2}{\pi m} \frac{1}{\left| m^2 - \frac{4l_y^2 - \lambda^2}{\lambda^2} \frac{l_z^2}{l_y^2} \right|} =
G^2 \sum \limits_{m=M+1}^{\infty} \frac{2}{\pi m} \frac{1}{\left| m^2 - G^2 \right|}$

Здесь $G = \frac{\sqrt{4l_y^2 - \lambda^2}}{\lambda} \frac{l_z}{l_y}$. Это константа, так что найдётся такое $M$, что $\forall m > M \colon m > G$. Будем всегда брать $M$ именно таким. Это позволит нам раскрыть последний модуль.

$\sum \limits_{m=M+1}^{\infty} \left| \frac{2\pi c^2}{m(w_m^2 - k^2)} \frac{4l_y^2 - \lambda^2}{l_y^2\lambda^2} \right| \leq G^2 \sum \limits_{m=M+1}^{\infty} \frac{2}{\pi m} \frac{1}{m^2 - G^2} \leq
G^2 \int \limits_M^{\infty} \frac{2}{\pi x} \frac{\diff x}{x^2 - G^2} =
G^2 \int \limits_M^{\infty} \frac{2}{\pi x} \frac{\diff x}{(x - G) (x + G)} \leq \\
\leq G^2 \int \limits_M^{\infty} \frac{2}{\pi x} \frac{\diff x}{(x + G)} \leq
G^2 \int \limits_M^{\infty} \frac{2 \diff x}{\pi x^2} =
\left. - G^2 \frac{2}{\pi x} \right|_M^{\infty} = \frac{2 G^2}{\pi M}$

Помня об указанных выше предположениях, оценим второй и третий хвосты.

$\sum \limits_{m=M+1}^{\infty} \left| \frac{2 k}{\pi m w_m} \right| =
\sum \limits_{m=M+1}^{\infty} \left| \frac{4 \pi c}{m \lambda \pi^2 c \sqrt{\frac{1}{l_y^2} + \frac{m^2}{l_z^2}}} \right| \leq
\sum \limits_{m=M+1}^{\infty} \left| \frac{4}{m \lambda \pi \sqrt{\frac{m^2}{l_z^2}}} \right| =
\sum \limits_{m=M+1}^{\infty} \frac{4 l_z}{\pi \lambda m^2} \leq \\
\leq \int \limits_M^{\infty} \frac{4 l_z \diff x}{\pi \lambda x^2} =
\left. - \frac{4 l_z}{\pi \lambda x} \right|_M^{\infty} =
\frac{4 l_z}{\pi \lambda M}$

$\sum \limits_{m=M+1}^{\infty} \left| \frac{k D}{w_m} \right| =
\sum \limits_{m=M+1}^{\infty} \left| \frac{2 \pi k c^2}{m w_m (w_m^2 - k^2)} \frac{4l_y^2 - \lambda^2}{l_y^2\lambda^2} \right| =
\frac{4l_y^2 - \lambda^2}{l_y^2\lambda^2} \sum \limits_{m=M+1}^{\infty} \left| \frac{4 \pi^2 c^3}{m \lambda \pi c \sqrt{\frac{1}{l_y^2} + \frac{m^2}{l_z^2}} \left( \pi^2 c^2 \left( \frac{1}{l_y^2} + \frac{m^2}{l_z^2} \right) - \frac{4 \pi^2 c^2}{\lambda^2} \right) } \right| \leq \\
\leq \frac{4l_y^2 - \lambda^2}{l_y^2\lambda^2} \sum \limits_{m=M+1}^{\infty} \left| \frac{4 \pi c^2}{m \lambda \sqrt{\frac{m^2}{l_z^2}} \left( \frac{\pi^2 c^2}{l_y^2} + \frac{\pi^2 c^2 m^2}{l_z^2} - \frac{4 \pi^2 c^2}{\lambda^2} \right) } \right| =
\frac{4l_y^2 - \lambda^2}{l_y^2\lambda^2} \sum \limits_{m=M+1}^{\infty} \left| \frac{4 \pi l_z c^2}{m^2 \lambda \pi^2 c^2 \left( \frac{1}{l_y^2} + \frac{m^2}{l_z^2} - \frac{4}{\lambda^2} \right) } \right| = \\
= \frac{4l_y^2 - \lambda^2}{l_y^2\lambda^2} \sum \limits_{m=M+1}^{\infty}  \frac{4 l_z}{m^2 \lambda \pi \left| \frac{1}{l_y^2} + \frac{m^2}{l_z^2} - \frac{4}{\lambda^2} \right|} =
\frac{4l_y^2 - \lambda^2}{l_y^2\lambda^2} \sum \limits_{m=M+1}^{\infty}  \frac{4 \lambda^2 l_y^2 l_z^3}{m^2 \lambda \pi \left| m^2 \lambda^2 l_y^2 - 4 l_y^2 l_z^2 + \lambda^2 l_z^2 \right|} = \\
= \frac{4l_y^2 - \lambda^2}{l_y^2\lambda^2} \sum \limits_{m=M+1}^{\infty}  \frac{4 l_z^3}{m^2 \lambda \pi \left| m^2 - \frac{4 l_y^2 l_z^2 - \lambda^2 l_z^2}{\lambda^2 l_y^2} \right|} =
\frac{4l_y^2 - \lambda^2}{\lambda^2} \frac{l_z^2}{l_y^2} \sum \limits_{m=M+1}^{\infty} \frac{4 l_z}{m^2 \lambda \pi \left| m^2 - \frac{4l_y^2 - \lambda^2}{\lambda^2} \frac{l_z^2}{l_y^2} \right|} = \\
= G^2 \sum \limits_{m=M+1}^{\infty} \frac{4 l_z}{m^2 \lambda \pi \left( m^2 - G^2 \right)} \leq
G^2 \int \limits_M^{\infty} \frac{4 l_z \diff x}{\pi \lambda x^2 \left( x^2 - G^2 \right)} = 
G^2 \int \limits_M^{\infty} \frac{4 l_z \diff x}{\pi \lambda x^2 (x - G) (x + g)} \leq
G^2 \int \limits_M^{\infty} \frac{4 l_z \diff x}{\pi \lambda x^2 (x + G)} \leq \\
\leq G^2 \int \limits_M^{\infty} \frac{4 l_z}{\pi \lambda x^3} =
\left. - G^2 \frac{8 l_z}{\pi \lambda x^2} \right|_M^{\infty} =
\frac{8 l_z G^2}{\pi \lambda M^2}$

Таким образом получили оценку хвоста: \\
$\left| r_{M+1} \right| \leq \frac{2 G^2}{\pi M} + \frac{4 l_z}{\pi \lambda M} + \frac{8 l_z G^2}{\pi \lambda M^2} = \frac{2 \lambda G^2 M + 4 l_z M + 8 l_z G^2}{\pi \lambda M^2}$

Это значение мы будем выдавать за ошибку при вычислениях:\\
$\varepsilon = \frac{2 \lambda G^2 M + 4 l_z M + 8 l_z G^2}{\pi \lambda M^2}$

Задумаемся над тем, сколько элементов ряда нам нужно оставить, чтобы выдержать наперёд заданную погрешность $\varepsilon$.

$\frac{2 \lambda G^2 M + 4 l_z M + 8 l_z G^2}{\pi \lambda M^2} - \varepsilon = 0$;

$\pi \lambda \varepsilon M^2 - 2 \left( \lambda G^2 + 2 l_z \right) M - 8 l_z G^2 = 0$.

Единственный положительный корень этого уравнения:

$M = \frac{2 \left( \lambda G^2 + 2 l_z \right) + \sqrt{4 \left( \lambda G^2 + 2 l_z \right)^2 + 32 \pi \lambda l_z \varepsilon G^2}}{2 \pi \lambda \varepsilon} =
\frac{\lambda G^2 + 2 l_z + \sqrt{\left( \lambda G^2 + 2 l_z \right)^2 + 8 \pi \lambda l_z \varepsilon G^2}}{\pi \lambda \varepsilon}$

Оценка $M$ при подсчёте округляется вверх до ближайшего целого.

\section{Исследование качества оценки остатка}
Исследуем с помощью ряда экспериментов качество полученной оценки с помощью следующего метода. С помощью полученной оценки найдем
теоретическое количество элементов ряда $N_t$, достаточное для выполнения данной оценки. Возьмем $0$ в качестве стабильного элемента. 
Будем идти с первого до $N_t$ элемента ряда, на каждом шаге проверяя, не различаются ли текущий элемент ряда и стабильный более, чем
на погрешность $\varepsilon$. Если различаются, то будем считать стабильным текущий элемент ряда. Искомым практическим количеством
элементов ряда, достаточным для удовлетворения заданной погрешности, будем считать номер последнего стабильного элемента.

Зафиксируем $y = \frac{l_y}{2}$. Для большей убедительности фиксируем $z$ в трех точках: $z_0 = \frac{l_z}{8}$, $z_1 = \frac{l_z}{4}$ 
и $z_2 = \frac{l_z}{2}$. Для верности время зафиксируем в достаточно большой момент $1$ с. Сведем результаты в таблицу 
\ref{tab:rem:experiment}, где $N_{p}^{(k)}$ соответствует количеству членов ряда при $z = z_k$.

\begin{table}[!hbtp]
  \centering
  \caption{Исследование качества оценки остатка ряда}
  \begin{tabular}{|c|r|r|r|r|r|}
    \hline
    $\mathrm{\varepsilon}$ & $0.1$ & $0.01$ & $0.001$ & $0.0001$ & $0.00001$ \\
    \hline
    $\mathrm{N_p}$ & $2705$ & $26713$ & $266792$ & $2667475$ & $26674407$ \\
    \hline
    $\mathrm{N_t^{(0)}}$ & $23$ & $60$ & $197$ & $1641$ & $7780$ \\
    \hline
    $\mathrm{N_t^{(1)}}$ & $22$ & $45$ & $210$ & $1127$ & $5334$ \\
    \hline
    $\mathrm{N_t^{(2)}}$ & $21$ & $63$ & $297$ & $1193$ & $6631$ \\
    \hline
  \end{tabular}
  \label{tab:rem:experiment}
\end{table}

Как видно, во всех трех точках разница между теоретической и практической оценками количества элементов ряда исчисляется порядками,
что позволяет сделать вывод о плохом качестве полученной оценки.
