\chapter*{Заключение}
\addcontentsline{toc}{chapter}{\tocsecindent{Заключение}}

В процессе выполнения работы была построена математическая модель физического процесса
распространения электромагнитной волны в волноводе, представляющая собой краевую задачу.
Было получено решение данной задачи в виде ряда Фурье по синусам. С помощью интегрального признака
Коши-Маклорена (см. \cite{sendov}) была исследована сходимость и получена грубая, но 
достоверная оценка остатка этого ряда. Проведено экспериментальное исследование качества такой оценки.
Это позволяет нам сделать следующие выводы.

Во-первых, метод разделения переменных Фурье применим в некоторых задачах, в которых его применение казалось
невозможным, с помощью дополнительных преобразований.\\
Во-вторых, метод Фурье не дает гарантии, что полученный ряд будет быстро сходиться. Как было видно при проведении
исследований, решение в виде ряда сходилось достаточно медленно, и, более того, полученная оценка погрешности
превышала реальную на порядки. Это позволило нам сделать вывод о грубости оценки.

