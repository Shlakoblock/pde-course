\chapter*{\begin{center}Задание\end{center}}

\begin{enumerate}
\item Осуществить математическую постановку краевой задачи
  для физического процесса, описанного в предложенном варианте
  курсовой работы.
\item Используя метод разделения переменных (Метод Фурье)
  получить решение задачи в виде ряда Фурье.
\item Исследовать сходимость ряда, получить оценку
  остатка ряда.
\item Разработать компьютерную программу расчета решения задачи
  (суммирования ряда Фурье) с требуемой точностью. При расчете
  коэффициентов ряда использовать метод численного интегрирования,
  если это необходимо. Обеспечить контроль погрешности численного
  интегрирования. Если необходимо, то разработать специальный программный
  модуль для расчета используемых собственных чисел оператора Лапласа.
  Обеспечить контроль погрешности расчета собственных чисел.
\item Используя разработанную программу провести экспериментальное
  исследование качества полученной аналитической оценки остатка ряда.
\item Оформить пояснительную записку к курсовой работе.
\end{enumerate}

\chapter*{\begin{center}Вариант №4\end{center}}
Разработать программу численного моделирования процесса
распространения электромагнитной волны в плоском однородном слое
толщиной $l_y$, шириной $l_z$, предполагая, что длина его бесконечна
$l_x = \infty$.\\

Рассмотреть случай такой поляризации волны, когда напряженности
электрического $\vec{E}$ и магнитного $\vec{H}$ полей имеют вид
\[
\vec{E} = (E_x, 0, 0);\qquad\vec{H} = (0, H_y, H_z).
\]

Записать краевую задачу для напряженности $E_x = E_x(y, z, t)$,
предполагая, что в начальный момент времени $t = 0$ при $0 < z \le
l_z, 0 \le y \le l_y, -\infty < x < +\infty$ среда находилась в
невозмущенном состоянии, а грани слоя $y = 0, y = l_y, z = l_z$
выполнены из электропроводящего материала. На грань $z = 0$ при $0 \le
t \le T$ подается возмущающая электромагнитная волна с напряженностью
\[
E_x(y, z=0, t) = \psi(y) \sin \frac{2\pi c}{\lambda}t,
\]
где $c$~--- скорость распространения волны в среде, $\lambda$~---
длина возмущающей волны.
\\
В качестве неизвестной функции использовать $\psi(y) = \sin\frac{\pi y}{l_y}$

Значения параметров:
\begin{eqnarray*}
  l_y &=& 2\cdot 10^{-5}\mbox{м};\\
  l_z &=& 2\cdot 10^{-5}\mbox{м};\\
  c &=& 3\cdot 10^{8}\mbox{м/с};\\
  \lambda &=& 2\cdot 10^{-6}\mbox{м};
\end{eqnarray*}
